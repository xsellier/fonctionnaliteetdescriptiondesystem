
\section{Question 5}

$$X_p \subseteq \lbrace H(p) \vert H : E \leftrightarrow C \wedge \bigwedge_{q \in E} H(q) \in X_q \rbrace$$

La formule $\mathcal{C'}$($\pi_0$) signifie qu'on applique les contraintes a chacune de nos classes d'\'equivalences.
Autrement dit, lorsque q et p seront d\'ependants, cela signifiera automatiquement que q et p seront dans la m\^eme classe d'\'equivalence.  De plus $X_p$ est inclu dans $\mathcal{C'}$($\pi_0$), donc pour chaque solution de $\mathcal{C}$($\pi_0$) nous aurons les m\^emes solutions que pour $\mathcal{C'}$($\pi_0$). Ce qui veut dire que si pour $\mathcal{C}$($\pi_0$) on a $\pi$(p) = $\emptyset$ qui implique $\pi$(q) = $\emptyset$ pour toute position de q $\in$ P, et qu'aucune grille n'est solution d'un remplissage initial I tel que $\pi_I \subseteq \pi_0$, alors cela est vrai pour $\mathcal{C'}$($\pi_0$).

\bigskip
