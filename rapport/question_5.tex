
\section{Question 5}

$$X_p \subseteq \lbrace H(p) \vert H : E \leftrightarrow C \wedge \bigwedge_{q \in E} H(q) \in X_q \rbrace$$

La formule $\mathcal{C'}$($\pi_0$) signifie qu'on applique les contraintes a chacune de nos classes d'\'equivalences.
Autrement dit, lorsque q et p seront d\'ependants, cela signifiera automatiquement que q et p seront dans la m\^eme classe d'\'equivalence.  De plus $X_p$ est inclu dans $\mathcal{C'}$($\pi_0$), donc pour chaque solution de $\mathcal{C}$($\pi_0$) nous aurons les m\^emes solutions que pour $\mathcal{C'}$($\pi_0$). Ce qui veut dire que si pour $\mathcal{C}$($\pi_0$) on a $\pi$(p) = $\emptyset$ qui implique $\pi$(q) = $\emptyset$ pour toute position de q $\in$ P, et qu'aucune grille n'est solution d'un remplissage initial I tel que $\pi_I \subseteq \pi_0$, alors cela est vrai pour $\mathcal{C'}$($\pi_0$).

\bigskip
Avec notre syst\`eme de contrainte $\mathcal{C'}$($\pi_0$) nous raffinons a chaque fois un peu plus notre grille. Pour une grille correctement initialis\'ee, tout comme avec $\mathcal{C}$($\pi_0$) nous obtiendrons une seule couleur par case ou alors un ensemble de couleur sur une case. La solution trouv\'e par $\mathcal{C'}$($\pi_0$) implique de n'avoir qu'une seule couleur par case et donc qu'une seule couleur par classe d'\'equivalence. Ce qui r\'epond parfaitement aux contraintes de $\mathcal{C}$($\pi_I$).

\bigskip
D'apr\`es nos contraintes $\mathcal{C'}$($\pi_0$) , la construction de notre grille solution sera toujours la m\^eme, et donc unique, tout comme $\mathcal{C}$($\pi_0$) n'a qu'une seule solution maximale. si on a une grille solution en utilisant $\mathcal{C'}$($\pi_0$), alors elle sera maximale. Car si on r\'eapplique nos contraintes, on ne modifiera pas les positions qui n'ont qu'une seule couleur et qui respectent les r\`egles du sudoku.