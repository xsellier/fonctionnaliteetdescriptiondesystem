
\section{Question 1}

$\exists$ p, q $\in$ P, ( G(p) = G(q) $\wedge$ q $\neq$ p )

Montrons que si ( G(p) = G(q) $\wedge$ q $\neq$ p ) $\Rightarrow$ q $\neg \sim_{l}$ p par l'absurde.

\begin{itemize}
\item On suppose que q $\sim_{l}$ p :
	\begin{itemize}
	\item p = ($i_p$; $j_p$) et q = ($i_q$; $j_q$)\\
	D'apr\`es $\sim_{l}$  on a :
		\begin{itemize}
		\item G(p) = G(q)
		\item $\underbrace{(i_p; j_p) \sim_{l} (i_q; j_q) \Rightarrow i_p = i_q}$\\
		$j_p$ = $j_q$ d'apr\`es l'\'enonc\'e.\\
		On en d\'eduit que q = p, ce qui contredit nos hypoth\`eses. \\
		Donc p ne peut pas etre en relation $\sim_{l}$ avec q.
		\end{itemize}
	\end{itemize}
\end{itemize}

\bigskip 
Montrons que si ( G(p) = G(q) $\wedge$ q $\neq$ p ) $\Rightarrow$ q $\neg \sim_{c}$ p par l'absurde.

\begin{itemize}
\item On suppose que q $\sim_{c}$ p :
	\begin{itemize}
	\item p = ($i_p$; $j_p$) et q = ($i_q$; $j_q$)\\
	D'apr\`es $\sim_{c}$  on a :
		\begin{itemize}
		\item G(p) = G(q)
		\item $\underbrace{(i_p; j_p) \sim_{l} (i_q; j_q) \Rightarrow j_p = j_q}$\\
		$i_p$ = $i_q$ d'apr\`es l'\'enonc\'e.\\
		On en d\'eduit que q = p, ce qui contredit nos hypoth\`eses. \\
		Donc p ne peut pas etre en relation $\sim_{c}$ avec q.
		\end{itemize}
	\end{itemize}
\end{itemize}


\bigskip
Montrons que si ( G(p) = G(q) $\wedge$ q $\neq$ p ) $\Rightarrow$ q $\neg \sim_{b}$ p par l'absurde.\\
D'apr\`es les r\`egles pr\'ec\'edentes on a $i_p$ $\neq$ $i_q$ et $j_p$ $\neq$ $j_q$. Mais vu que p et q se trouvent dans le m\^eme carr\'e on a :\\
0 $\leq$ | $i_q$ - $i_p$ | < d et\\
0 $\leq$ | $j_q$ - $j_p$ | < d\\

$\exists$ r, s $\in$ [0, d-1] tels que :\\
$$
\left\{
    \begin{array}{ll}
        r \times d \leq i_p, i_q < (r + 1) \times d\\
        s \times d \leq j_p, j_q < (r + 1) \times d
    \end{array}
\right.
$$

donc 

$$
\left\{
    \begin{array}{llll}
        r \times d \leq i_p < d^2, 0 \leq i_q < (r + 1) \times d\\
        s \times d \leq j_p < d^2, 0 \leq j_q < (s + 1) \times d\\
        0 \leq | i_q - i_p | < d\\
		0 \leq | j_q - j_p | < d
    \end{array}
\right.
$$

\bigskip
D'ap\`es les r\`egles de math\'ematiques vues en seconde on obtient :\\
$$
\left\{
    \begin{array}{llll}
        r \times d \leq i_p < d^2, 0 \leq i_q < (r + 1) \times d\\
        s \times d \leq j_p < d^2, 0 \leq j_q < (s + 1) \times d\\
        0 \leq | i_q - i_p | \leq 0 $ $ (car $ $ d^2 - (r + 1) \times d \leq 0 )\\
		0 \leq | j_q - j_p | \leq 0 $ $ (car $ $ d^2 - (s + 1) \times d \leq 0 )
    \end{array}
\right.
$$

\bigskip
On en d\'eduit que $i_p$ = $i_q$ et que $j_p$ = $j_q$ ce qui implique que p = q.\\
Ce qui contredit nos hypoth\`eses. Donc p ne peut pas etre en relation $\sim_{b}$ avec q.