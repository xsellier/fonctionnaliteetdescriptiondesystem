
\section{Question 3}

Soit p une position dans notre grille ne poss\'edant qu'une seule couleur. Soit q, une position d\'ependante et diff\'erente de p. 

Soit C' l'ensemble des couleurs disponibles pour q. Apr\`es le passage de la fonction f(X), on associera \`a q l'ensemble C' $\setminus \lbrace c_p \rbrace$. 

\bigskip
En clair ce mode de r\'esolution s\'equentiel s'applique de mani\`ere unique a chaque position q d\'ependante de chaque position p.

Si maintenant nous obtenons une solution maximale respectant le syst\`eme de contrainte $\mathcal{C}$($\pi_0$), alors en r\'eappliquant une fois de plus nos contraintes, cela ne changerait rien, donc on aurait un $\mathcal{C}$($\pi_{0'}$) identique a $\mathcal{C}$($\pi_0$). 

Par cons\'equent notre solution maximale est unique.