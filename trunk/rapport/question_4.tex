
\section{Question 4}

\bigskip
\begin{itemize}
\item Si $\pi$(p) = $\emptyset$ cela implique que toutes les positions q d\'ependantes de p sont aussi $\emptyset$. 

Car d'apr\`es nos contraintes, a chaque fois que l'on retire une couleur \`a l'ensemble de couleur associ\'e \`a la position p, alors on la retire aussi a chaque ensemble de couleurs des positions q, si elles sont d\'ependantes de p. 

Et tout cela d'apr\`es notre fonction $\mathtt{f}$(X). 

Donc nous aurons les m\^emes couleurs pour p et chacune des positions q d\'ependantes de p.

\bigskip
On en d\'eduit donc que $\pi(q)=\emptyset$ pour toute position q $\in$ P.

\bigskip
\item Si $\pi$(p) = $\emptyset$ alors, d'apr\`es (1) $\pi$(q) = $\emptyset$ pour q $\in$ P. 

Autrement dit notre raffinement $\pi \subseteq \pi'$. Par cons\'equent Nous obtenons des cases sans aucune couleur et donc impossible a compl\'eter. 

\bigskip
On en d\'eduit facilement que dans ces conditions nous ne pouvons obtenir de grille solution d'un remplissage initial I tel que $\pi_I \subseteq \pi_0$.
\end{itemize}