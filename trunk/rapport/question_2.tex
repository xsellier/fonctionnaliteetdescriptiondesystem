
\section{Question 2}

Lorsque notre grille est initialis\'ee elle respecte D = {(p, q) $\in$ P $\times$ P | p $\neq$ q $\wedge$ ( $\sim_{l}$ $\vee$ $\sim_{c}$ $\vee$ $\sim_{b}$ )}

En regardant de plus pr\`es $\mathcal{C}$($\pi_I$) on distingue bien deux cas :

\bigskip
\begin{itemize}
\item $X_p \subseteq \pi_0(p)$ pour chaque p $\in$ P

Cette premi\`ere proposition va nous servir a raffiner notre grille. Pour chaque position p, si une couleur unique y est associ\'ee alors $\pi_G$(p) = {G(p)}.


Dans l'autre cas $\pi_I$(p)  nous ne sommes pas sur que p ne contiennent qu'une seule couleur.  

\bigskip
Par cons\'equent on teste voir le nombre de couleur possible pour la position p. Si on a affaire a un couple {(p, c)} alors on attribue le singleton {c} \`a la position p. dans le cas contraire nous laissons l'ensemble C pour la position p.

\bigskip
\item $X_q \subseteq \mathtt{f} (X_p)$ pour chaque (p, q) $\in$ D

Dans ce cas on applique $\mathtt{f} (X_p)$ pour p et q qui sont d\'ependants. Si p a d\'ej\`a une seule couleur attribu\'ee, alors on la retire pour toutes les autres positions d\'ependantes de p si elles y existent. Si jamais p, n'a pas une seule couleur, alors on ne modifie pas les possibilit\'es. 

\bigskip
On effectue ce travail sur chacune des positions p, et pour chacune de ces positions on l'effectue en rapport avec des positions q d\'ependantes de p.

Donc la valuation $\pi_G$ satisfait le syst\`eme de contraintes $\mathcal{C}$($\pi_0$)
\end{itemize}